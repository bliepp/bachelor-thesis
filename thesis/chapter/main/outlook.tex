\chapter{Zusammenfassung und Ausblick}



Ziel der Arbeit war die molekulardynamische Simulation von selektiven Laserschmelzen, einem
Verfahren der additiven Fertigung, bei dem ein Laser eine Metallpulverschicht nach der anderen
selektiv zum Schmelzen bringt und so das gewünschte Objekt formt. Zu beachten gibt es hierbei,
dass die Pulverpartikel Durchmesser in der Größenordnung weniger Mikrometer haben, die dafür
notwendige Teilchenzahl allerdings selbst mit den aktuell gängigen Höchstleistungsrechnern nur
schwer realisierbar ist. Während Simulationen mesoskopischer Größenordnung schon gemacht wurden
\cite{panwisawas2017mesoscale}, wird hier ein Skalierungsansatz genommen. Das ganze Modell wird
dafür kleiner skaliert (Durchmesser bei wenigen hundert \AA ngström) und auf Validität untersucht.

Im ersten von zwei Teilen werden zuerst die ingenieurwissenschaftliche Datenlage und die
Grundlagen der Molekulardynamik besprochen. Besonders wichtig sind hierbei die beim selektivem
Laserschmelzen auftretenden Defekte sowie die notwendigen Näherungen bei der Entwicklung des
Modells.

Im zweiten Teil werden dann die Simulationsergebnisse ausgewertet und besprochen. Hier geht es
überwiegend darum zu argumentieren, ob dieses Modell geeignet zur Beschreibung von selektivem
Laserschmelzen.

Tatsächlich finden sich Hinweise auf die den Defekten zugrundeliegenden Mechanismen. Zwar ist
das Modell, welches das Ergebnis dieser Arbeit war, bei weitem noch nicht vollständig, aber die
grundlegenden Voraussetzungen scheinen erfüllt zu sein, um ein fundiertes Modell abzugeben.
So konnte beispielsweise schon ein Mechanismus zur Porenbildung identifiziert werden, wobei diese
Poren durch das Abhanden-sein eines Umgebungsgases nicht erhalten geblieben sind. Außerdem konnten
gezeigt werden, dass eine langsamere Lasergeschwindigkeit die Tröpfchenbildung begünstigt, was
Sadali et. al. als eine mögliche Ursache für Defekte wie balling oder splashing benannten
\cite{sadali2020influence}.

Die Arbeit verwendet hierbei reines Aluminium im Vakuum. Zwar sind reine Metalle im SLM
eher die Seltenheit, jedoch besteht das oft verwendete AluSi10Mg beispielsweise aus ungefähr
\SI{90}{\percent} Aluminium. Dazu kommt, dass aufgrund der vielen unbekannten Sachverhalte
bei diesem Thema viele offene Fragen existieren. Um die Zahl der bekannten Faktoren in diesem
stark simplifizierte Modell zu erhöhen, wurde vorerst auf Aluminium gesetzt, da hierzu zahlreiche
Abschlussarbeiten des Instituts vorliegen.

Das Modell ist natürlich noch lange nicht vollständig. Während Einzelaspekte und Indizien schon
erkennbar sind bedarf es für eine realistischere Simulation weiterer Verbesserungen. So ist die
Einbettung eines umgebenden Schutzgases wie beispielsweise Argon notwendig, um Defekte wie Poren
oder Mikrorisse akkurat simulieren zu können.

Auch eine Erweiterung um eine simulierte Kühlungseinheit könnte hilfreich für die Genauigkeit
der Ergebnisse sein. Bis jetzt ist das geschmolzene Material ja nicht in der Lage wieder zu seiner
üblichen Gitterform zu erstarren. Die perfekte Isolierung nach den Regeln des NVE-Ensembles führt
auch zu dem Trend selbst lange nach der Bestrahlung durch einen Laser noch weiteres Material zu
schmelzen. Dem könnte beispielsweise mit einer simulierten Aktivkühlung entgegengewirkt werden.

Trotz dem hohen Erweiterungspotential des Modells ist es dennoch schon benutzbar. So können jetzt
schon nach dem gleichen Prinzip unterschiedliche Proben untersucht werden. Diese könnten
beispielsweise Aneinanderreihungen von mehreren Kugeln sein, um deren gegenseitige Einflussnahme
besser darzustellen. Ebenfalls sind geschichtete Kugeln, eventuell sogar unterschiedlicher Größe,
von Interesse. Bekanntermaßen ist eines der häufigsten Ursachen für Porenbildung eine mangelnde
Packungsdichte des Materialpulvers. Es wäre von höchstem Interesse die mikroskopische Porenbildung
für verschiedene Packungsdichten zu Untersuchen. Die Simulation mehrerer Partikel liegt nicht nur
offensichtlicher Weise näher an der Realität, sondern macht auch erst das Auftreten diverser
Effekte möglich. Ebenfalls ist zu beachten, dass in dieser Arbeit eine Kugel auf festem Grund
betrachtet wurde. Da in der meisten Zeit das ungeschmolzene Pulver aber über eines bereits
verarbeiteten Schicht liegt, ist die Wechselwirkung mit einem nicht-fixierten Boden ebenfalls
nicht unwichtig.
