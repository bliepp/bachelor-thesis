\chapter{Theoretische Einführung}



\section{Grundlagen des 3D-Drucks mittels selektivem Laserschmelzen (SLM)}
    Die additive Fertigung (AM) spielt in unserem Leben eine immer größere Rolle. Während sich bereits
    ein Hobbymarkt für preisgünstige 3D-Drucker im eigenen Haushalt etabliert hat, ist die
    industrielle Anwendung nicht zu vernachlässigen. Eine von Essentium, einem Hersteller von
    industriellen 3D-Druckern und Materialien, angefertigte Umfrage zeigte, dass sich die Anzahl der
    Firmen, die 3D-Druck zur industriellen Fertigung benutzen, von 2018 auf 2019 fast verdoppelte.
    Während der Anteil 2018 noch bei 21\% lag, änderte er sich innerhalb eines Jahres auf 40\%
    \cite{stevenson2019survey}.
    
    Obwohl es eine Vielzahl unterschiedlicher Methoden der additiven Fertigung gibt, funktionieren
    allen Methoden nach einem ähnlichen Grundprinzip: Das digitale 3-dimension\-ale Objekt wird in
    Schnittebenen unterteilt, die vom 3D-Drucker nacheinander produziert und aufgeschichtet werden.

    \subsection{Funktionsweise und Unterschiede zu anderen Methoden}
        Das vermutlich bekannteste Verfahren zur additiven Fertigung ist wahrscheinlich das
        Fused Depositing Modeling (FDM), aus markenrechtsgründen auch bekannt als
        Fused Filament Fabrication (FFF). Dabei wird ein Kunststofffilament aufgeschmolzen und durch
        eine Düse extrudiert. Die Düse fährt hierbei die Schnittebene ab, sodass Ebene für Ebene das
        gewünschte Objekt aufgeschichtet wird. Schichtdicken sind bei diesem Verfahren üblicherweise
        zwischen \SI{0,025}{\milli\meter} und \SI{1,25}{\milli\meter} \cite{wikipedia2021fused}.

        Obwohl dieses Verfahren relativ einfach umzusetzen ist wird es durch verschiedene Faktoren
        limitiert. So ist beispielsweise für die Qualität die Form des zu druckenden Objekts relevant.
        Außerdem ist dieses Verfahren auf wenige Materialien beschränkt, wie Thermoplaste und Formwachse
        \cite{wikipedia2021fused}.
        \todo[color=red]{3D-Druck Buch zuhause zitieren}

    \subsection{Einfluss der Lasergeschwindigkeit auf die Oberflächenbeschaffenheit}
    \subsection{Beobachtbare Defekte im Material}


\section{Molekulardynamische Modellierung von SLM}
    \subsection{Grundlegende Funktionsweise der Molekulardynamik (MD)}
    \subsection{Notwendige Näherungen und Vereinfachungen}
        \todo[color=red]{Kleinere Skala und größere Gravitation mithilfe von \cite{glosli2007extending} motivieren.}
    \subsection{Implementierung des Laser}
    \subsection{Besonderheiten bei der Simulationssoftware IMD}