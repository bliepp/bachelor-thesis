\chapter{Theoretische Einführung}



\section{Grundlagen des 3D-Drucks mittels selektivem Laserschmelzen (SLM)}
	Die \emph{additive Fertigung} (AM) spielt in unserem Leben eine immer größere Rolle. Während
	sich bereits ein Hobbymarkt für preisgünstige 3D-Drucker im eigenen Haushalt etabliert hat,
	ist die industrielle Anwendung nicht zu vernachlässigen. Eine von Essentium, einem Hersteller
	von industriellen 3D-Druckern und Materialien, angefertigte Umfrage zeigte, dass sich die
	Anzahl der Firmen, die 3D-Druck zur industriellen Fertigung benutzen, von 2018 auf 2019 fast
	verdoppelte. Während der Anteil 2018 noch bei 21\% lag, änderte er sich innerhalb eines Jahres
	auf 40\% \cite{stevenson2019survey}.

	Obwohl es eine Vielzahl unterschiedlicher Methoden der additiven Fertigung gibt, funktionieren
	alle Methoden nach einem ähnlichen Grundprinzip: Das digitale 3-dimension\-ale Objekt wird
	durch einen sogenannten \emph{Slicer} in Schnittebenen unterteilt, die vom 3D-Drucker
	nacheinander produziert und aufgeschichtet werden.

	%\subsection{Funktionsweise und Unterschiede zu anderen Methoden}
	\subsection{Funktionsweise und Unterschiede zum bekannteren FDM-Druck}
		Das vermutlich bekannteste Verfahren zur additiven Fertigung ist wahrscheinlich das
		\emph{Fused Depositing Modeling} (FDM), aus markenrechtsgründen auch bekannt als
		\emph{Fused Filament Fabrication} (FFF). Dabei wird ein Kunststofffilament aufgeschmolzen
		und durch eine Düse extrudiert. Die Düse (\emph{Hotend}) fährt hierbei während des
		extrudierens wie in Abbildung \ref{fig:fdm} dargestellt die Schnittebene ab, sodass
		Ebene für Ebene das gewünschte Objekt aufgeschichtet wird. Schichtdicken sind bei diesem
		Verfahren üblicherweise zwischen \SI{0,025}{\milli\meter} und \SI{1,25}{\milli\meter}
		\cite{wikipedia2021fused}.

		Obwohl dieses Verfahren relativ einfach umzusetzen ist wird es durch verschiedene Faktoren
		limitiert. So ist beispielsweise für die Qualität die Form des zu druckenden Objekts
		relevant. Außerdem ist dieses Verfahren auf bestimmte Materialien beschränkt, wie
		Thermoplaste und Formwachse \cite{wikipedia2021fused}.

		\begin{figure}[ht]
			\centering
			\includegraphics[width=0.9\textwidth]{chapter/main/img/fdm.png}
			\caption[Schematische Darstellung des FDM-/FFF-Verfahrens]{Schematische Darstellung
			einer möglichen Konfiguration des als FDM oder FFF bekannten 3D-Druck-Verfahrens. Ein
			Fördermechanismus schiebt das Filament durch eine aufgeheizte Düse. Durch eine
			Bewegung der Düse und/oder des Druckbetts in mindestens 3 Achsen erfolgt eine Formung
			des gewünschten Objektes Ebene für Ebene. \cite[S. 114]{horsch20143d}}
			\label{fig:fdm}
		\end{figure}

		Das in dieser Arbeit beobachtete Verfahren ist jedoch das \emph{selektive Laserschmelzen}
		(SLM). Der Drucker besteht hierbei aus zwei Gefäßen mit beweglichen Böden (Abbildung
		\ref{fig:slm_sls}). Das Gefäß für den Materialvorrat (\emph{feed container}) ist mit dem
		Material in Pulverform gefüllt, während das andere wenig bis gar nicht gefüllt ist. Beim
		Drucken einer neuen Ebene hebt sich der Boden des Materialvorrats und eine Walze
		(\emph{Beschichtungseinheit}) überträgt das dosierte Material in den Druckraum, dessen
		Druckbett abgesenkt wird. Ein beweglicher Laser fährt nun die gewünschte Schnittebene ab
		und verschmilzt das Pulver zu einer festen Form.

		\begin{figure}[ht]
			\centering
			\includegraphics[width=0.9\textwidth]{chapter/main/img/sls_slm.png}
			\caption[Schematische Darstellung des SLM-Verfahrens]{Schematische Darstellung des
			SLM-Verfahrens. Der bewegliche Spiegel sorgt für eine präzise Positionierung des
			Laser-Punktes. Während sich der Boden des Vorrats\-gefäßes im Prozess anhebt, senkt
			sich der Boden des Druckraums ab. \cite[S. 119]{horsch20143d}}
			\label{fig:slm_sls}
		\end{figure}

		Das Druckbett wird üblicherweise um \SI{30}{\micro\meter} bis \SI{100}{\micro\meter}
		abgesenkt \cite{song2012effects}. Zu den am häufigsten verwendete Materialien, die beim
		SLM-Verfahren verwendet werden, zählen Titanlegierungen, insbesondere das in der Luft- und
		Raumfahrt populäre Ti6Al4V \cite{song2012effects,shi2016performance,brandl2012morphology}. % maybe sadali2020influence, too? but it refs brandl2012morphology
		Aluminiumlegierungen werden ebenfalls verwendet
		\cite[je Al-Si-10Mg]{yan2020comparative,zou2017study}.

	\subsection{Beobachtbare Defekte im fertigen Bauteil}
		\subsubsection{Porenbildung}
		Aufgrund verschiedener physikalischer Phänomene treten während des Verfahrens Vorgänge auf,
		die sich in Materialdefekten bemerkbar machen. Einer der bekannstesten Defekte ist die
		\emph{Porenbildung}. Mit einer Größe von weniger als \SI{100}{\micro\meter} zählen diese
		Gaseinschlüsse zu den kleinsten der hier aufgeführten Defekte \cite{zhang2017defect}. Es
		werden zwei verschiedene Gasquellen unterschieden. Während eine Quelle des dafür
		notwendigen Gases die Verdampfung mancher Legierungsbestandteile, wie beispielweise
		Magnesium, sein kann, kann ein Gaseinschluss nicht zuletzt auch durch Einschluss des beim
		Prozess verwendeten Umgebungsgases erfolgen \cite{galy2018main}. In letzterem Fall kann
		zum Beispiel eine zu geringe Packungsdichte des verwendeten Materialpulvers die Ursache
		sein. Das Gas dringt hierbei während des Schmelzprozesses in die Schmelze ein, erreicht
		aber aufgrund einer hohen Kühlrate die Oberfläche nicht vor dem Erstarren, sodass beim
		Erhärten des Materials ein Lufteinschluss die Folge ist. Aus diesem Grund haben Poren eine
		näherungsweise sphärische Form. Sie sind üblicherweise im im gefertigten Bauteil
		gleichverteilt aufzufinden. Das komplette Eliminieren aller Poren gilt als sehr schwer
		\cite{zhang2017defect}. Galy et al. zufolge hat die Porengröße und -form einen
		entscheidenden Einfluss auf die Qualität des Bauteils. Größere Makroporen verschlechtern
		die Qualität stärker als kleinere Mikroporen, wobei sich mehrere Mikroporen während einer
		Hitzebehandlung zu größeren Makroporen vereinigen können. Eine stärkere Entrundung der
		sphärischen Form der Poren kann eine Ursache von Rissen im Bauteil sein
		\cite{galy2018main}. Poren unterschiedlicher Größe sind in Abbildung
		\ref{fig:defects_porosities} zu sehen.

		\begin{figure}[ht]
			\centering
			\includegraphics[width=0.7\textwidth]{chapter/main/img/defects/porosities.png}
			\caption{Poren sphärischer Form unterschiedlicher Größe \cite{zhang2017defect}.}
			\label{fig:defects_porosities}
		\end{figure}

		\begin{figure}[ht]
			\centering
			\includegraphics[width=0.7\textwidth]{chapter/main/img/defects/lack_of_fusion.png}
			\caption{Was auch immer \cite{zhang2017defect}}
			\label{fig:defects_lof}
		\end{figure}

		\begin{figure}[ht]
			\centering
			\includegraphics[width=0.7\textwidth]{chapter/main/img/defects/cracks.png}
			\caption{Was auch immer \cite{zhang2017defect}}
			\label{fig:defects_cracks}
		\end{figure}

	\subsection{Wichtige Parameter und Einfluss der Lasergeschwindigkeit}
		Beim SLM-Verfahren sind verschiedene Parameter einstellbar, die das Resultat in
		unterschiedlicher Weise und Stärke beeinflussen. Die wichtigsten Parameter hierfür sind,
		wie in Abbildung \ref{fig:slm_parameters} zu sehen, die Lasergeschwindigkeit
		(\emph{scanning speed}), die Laserleistung, der horizontale Bahnabstand
		(\emph{hatch distance}, \emph{hatching distance} oder \emph{hatch spacing}), die
		Wellenlänge des Lasers, der Durchmesser des fokussierten Laserpunktes
		(\emph{spot size}) und die Dicke der Pulverschicht (\emph{layer thickness})
		\cite{sadali2020influence}.

		\begin{figure}[ht]
			\centering
			\includegraphics[width=0.8\textwidth]{chapter/main/img/slm_parameters.jpg}
			\caption{Darstellung und räumliche Einordnung der wichtigsten einstellbaren Parameter
			beim SLM-Druck. \cite{saunders2017x}}
			\label{fig:slm_parameters}
		\end{figure}


\section{Molekulardynamische Modellierung von SLM}
	\subsection{Grundlegende Funktionsweise der Molekulardynamik (MD)}
	\subsection{Notwendige Näherungen und Vereinfachungen}
		\todo[color=red]{Kleinere Skala und größere Gravitation mithilfe von \cite{glosli2007extending} motivieren.}
	\subsection{Implementierung des Laser}
	\subsection{Besonderheiten bei der Simulationssoftware IMD}
